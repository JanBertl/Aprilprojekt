\documentclass[a4paper, 12 pt]{article}

\usepackage[x11names]{xcolor}
\usepackage[top=0.75in, bottom=0.2in, left=0.35in, right=0.35in]{geometry}
\usepackage{graphicx}
\usepackage{booktabs}
\usepackage{url}
\usepackage{enumitem}
\usepackage{palatino}
\usepackage{tabularx}

\newcommand{\coloredSectionDark}[1]{{\small \colorbox{DodgerBlue2}{\begin{minipage}{0.99\textwidth}{\textbf{#1 \vphantom{p\^{E}}}}\end{minipage}}}}
\newcommand{\coloredSection}[1]{{\small \colorbox{DeepSkyBlue1}{\begin{minipage}{0.99\textwidth}{\textbf{#1 \vphantom{p\^{E}}}}\end{minipage}}}}

\begin{document}
\section*{\textcolor{DodgerBlue2}{APRILPROJEKT - KUCHEN BACKEN}}

\noindent
\coloredSectionDark{\textbf{\textcolor{white}{PROJEKTDEFINITION}}}\\[-0.3cm]
\\\\
\coloredSection{\textbf{\textcolor{white}{AUSGANGSSITUATION}}}\\[-0.3cm]
\\
Egal ob ein Verwandter oder ein guter Freund 
auf Besuch kommt, Kaffee und Kuchen bietet sich immer an. Ein Kuchen ist in vielen Situationen gut zu gebrauchen.
Der Kuchen selbst ist dabei eine nette Geste oder auch eine gesellschaftliche Konvention, je nach Betrachtungsweise. Die Funktion
des Kuchens liegt letztlich darin, den schnellen Hunger zu stillen und damit die Stimmung etwas zu heben. Ebenfalls kann der Vorgang des 
Backens, je nach Person, Freude bereiten.\\\\
\noindent
\coloredSection{\textbf{\textcolor{white}{RISIKEN UND GEGENMA{\ss}NAHMEN}}}\\[-0.3cm]
\begin{table}[!ht]
    \begin{tabular*}{\columnwidth}{l|l}
        \hline
        \textbf{Risiken}       & \textbf{Gegenma{\ss}nahmen} \\ \hline 
        unvollständiges Rezept & genaue Recherche            \\ \hline 
        Mehl ist ausverkauft   & rechtzeitig Mehl auf Vorrat kaufen           \\ \hline 
        Eigelb vermischt sich mit Eiweiß & Eier sorgfältig und einzeln verarbeiten \\ \hline 
        Schalen könnnen kaputt gehen                      & Plastik- oder Metallschalen verwenden                           \\ \hline 
        Kuchen bleibt zu lange im Backofen        & Timer stellen                           \\ \hline 
    \end{tabular*}
\end{table}

\noindent
\coloredSection{\textbf{\textcolor{white}{ZUTATEN}}}\\[-0.3cm]
\\
\textbf{Teig:} 
\begin{itemize}
    \item 6 Eier
    \item 1 Prise Salz
    \item 150g Zucker
    \item 130g Mehl
    \item 30g Kakao
    \item 1 Prise Backpulver
\end{itemize} 
\textbf{Tränke:}
\begin{itemize}
    \item 5 EL Kirschwasser
    \item 100ml Wasser
\end{itemize} 
\textbf{Füllung:}
\begin{itemize}
    \item 1 Glas Sauerkirschen
    \item 20g Speisestärke
    \item 50g Zucker
\end{itemize} 
\textbf{Creme:}
\begin{itemize}
    \item 800ml Schlagsahne
    \item 3 Packungen Sahnesteif
    \item 80g Puderzucker
    \item 1 EL Vanillezucker
\end{itemize} 
\textbf{Fertigstellung:}
\begin{itemize}
    \item 200ml Schlagsahne
    \item 1 Packung Sahnesteif
    \item 100g Schokoraspel
    \item Belegkirschen
\end{itemize} \\
\end{document}