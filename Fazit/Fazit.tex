\documentclass[a4paper, 12 pt]{article}

\usepackage[x11names]{xcolor}
\usepackage[top=0.75in, bottom=0.2in, left=0.35in, right=0.35in]{geometry}
\usepackage{graphicx}
\usepackage{booktabs}
\usepackage{url}
\usepackage{enumitem}
\usepackage{palatino}
\usepackage{tabularx}

\newcommand{\coloredSectionDark}[1]{{\small \colorbox{DodgerBlue2}{\begin{minipage}{0.99\textwidth}{\textbf{#1 \vphantom{p\^{E}}}}\end{minipage}}}}
\newcommand{\coloredSection}[1]{{\small \colorbox{DeepSkyBlue1}{\begin{minipage}{0.99\textwidth}{\textbf{#1 \vphantom{p\^{E}}}}\end{minipage}}}}

\begin{document}
\section*{\textcolor{DodgerBlue2}{APRILPROJEKT - KUCHEN BACKEN}}

\noindent
\coloredSectionDark{\textbf{\textcolor{white}{FAZIT}}}\\[-0.3cm]
\\
Wenn man die richtigen Werkzeuge und Ressourcen zur Verfügung hat ist das Herstellen einer \\
Schwarzwälder Kirschtorte gar nicht so schwer.
Ebenfalls ist es deutlich billiger, als ihn im Supermarkt oder beim Bäcker zu kaufen. Andererseits benötigt die Herstellung etwas Geschick in der Küche
und natürlich Zeit. \\
Bei der Herstellung müssen die Risiken beachtet werden und man darf nicht von der vorgegebenen Anleitung abweichen. Dies haben wir umgesetzt, weshalb uns
die Herstellung des Kuchens erfolgreich gelungen ist.
\\\\\\
\end{document}